

ASTEC can run only on Linux system and was tested on:
\begin{itemize}
\item Ubuntu 14.04 64 bits
\end{itemize}



\section{Requirements}
In order to be able to compile the different source code you need to install several packages, mostly from the terminal application.
	    
\subsection{to run the different codes:}
\begin{itemize}
\item python 2.7 or higher  

Installation:
\begin{itemize}
\item Linux: Should be installed, refer with command line 

\texttt{`python -V`} 

to get the version or visit  \url{http://php.net}
\end{itemize}
\item and a C compiler

Installation
\begin{itemize}
\item Linux: Should be installed, refer with command line 

\texttt{`dpkg --list | grep compiler`} 

to get list of available C compiler version or install \texttt{gcc} with 

\texttt{`sudo apt-get install gcc`}
\end{itemize}
\end{itemize}


\subsection{the install libraries are necessary:}
\begin{itemize}
\item \texttt{pip}, an installer for python (\url{https://pypi.org/project/pip/}). 
Installation:
\begin{itemize}
\item Linux: run command line 

\texttt{`sudo apt-get install python-pip python-dev build-essential`}
\item Others: see \textit{Installation} from \url{https://pypi.org/project/pip/}
\end{itemize}
\item \texttt{cmake}, a software to build source code (\texttt{http://www.cmake.org})
\begin{itemize}
\item Linux: run command line 

\texttt{`sudo apt-get install cmake`}
\item Others: see \textit{Download} from \url{https://cmake.org/}
\end{itemize}
\end{itemize}




\subsection{the following python libraries are necessary:}
\begin{itemize}
\item numpy, scipy, matplotlib: different scientific packages for python  
(\url{http://www.numpy.org}, \url{http://www.scipy.org}, \url{http://matplotlib.org}).

Installation
\begin{itemize}
\item Linux: run command line 

\texttt{`sudo apt-get install python-numpy python-scipy python-matplotlib ipython ipython-notebook python-pandas python-sympy python-nose`}
\end{itemize}

\item jupyter notebook: a open-source framework to share scientific code (\texttt{http://jupyter.org})

Installation:
\begin{itemize}
\item Linux: run command line 

\texttt{`sudo pip install jupyter`}
\end{itemize}

 \item libhdf5-dev,cython,h5py  a library to read/write hdf5 images (\url{https://www.hdfgroup.org/HDF5/})
 
 Installation:
 \begin{itemize}
 \item Linux: run the command lines for a global installation
 
 \texttt{`sudo apt-get install libhdf5-dev`}
 
 \texttt{`sudo pip install cython`}
 
 \texttt{`sudo pip install h5py`}
 
 or 
 
  \texttt{`pip install --user cython`}
 
 \texttt{`pip install --user h5py`}
 
 \end{itemize}

 \item pylibtiff  a library to read/write tiff/tif images (\url{https://pypi.python.org/pypi/libtiff/}).
 
 Installation:
\begin{itemize} 
\item  You can install the pylibtiff distributed with ASTEC
 
\texttt{`cd path/to/Package/ASTEC/CommunFunctions/libtiff-0.4.0/; sudo python setup.py install`}

\item Alternatively:

\texttt{`sudo pip install libtiff`}

or 

\texttt{`pip install --user libtiff`}
\end{itemize}
\end{itemize}

\subsection{the following  libraries are necessary:}
\begin{itemize}
\item \texttt{zlib}, a compression library (\texttt{http://www.zlib.net})

Installation (if not already installed)
\begin{itemize}
\item Linux: run command line 

\texttt{`sudo apt-get install zlib1g-dev`}

\end{itemize}

\item \texttt{libtiff}, \url{http://libtiff.org}
				
\end{itemize}


\section{Installation}
      You need to clone the \texttt{morpheme-privat} project in your computer: run 
        
\texttt{`cd <morpheme-privat\_parent\_directory>`}

\texttt{`git clone git+ssh://<username>@scm.gforge.inria.fr/gitroot/morpheme-privat/morpheme-privat.git`}

        Please note that for this step, you may need to ask for an account on the inria gforge and for the access to the morpheme-privat project (for further information, contact \url{gregoire.malandain@inria.fr})
        
Then you need to compile the dependencies LEMON, NIFTICLIB, TIFF in \texttt{<morpheme-privat\_parent\_directory>/external}
by creating in each subrepository a build folder, going in it, and run:

\texttt{`ccmake ..`}

then configure, generate and quit (for NIFTICLIB, please take note that you should set the \texttt{CMAKE\_C\_FLAGS} to value '-fPIC' in the advanced mode ; you need therefore to press 't' to toogle advanced mode)

        run \texttt{`make -j<nb proc>`} where \texttt{<nb proc>} is the number of processor require to compile the code 
      Then go to \texttt{<morpheme-privat\_parent\_directory>/vt} directory and follow these instructions:

\texttt{`mkdir build`}

\texttt{`cd build`}

\texttt{`ccmake ..`}

Press 'c' to configure. At this step, set the links to the build folders of NIFTI, LEMON and TIFF. Press 'c' to re-configure. Press 'g' to generate and 'q' to quit.

Run \texttt{`make -j<nb proc>`} where \texttt{<nb proc>} is the number of processor require to compile the code 
			
\subsection{Previously}
	    Since the vt library is compiled, you need to create the symbolic link of the folder \texttt{<morpheme-privat\_parent\_directory>/vt} to \texttt{ASTEC/CommunFunctions/cpp}.
 in a terminal:
 
 \texttt{`cd path/to/Package/ASTEC/CommunFunctions`}
 
 \texttt{ln -s <complete\_path\_of\_vt\_folder> cpp}

\subsection{From version v2.0}

	    Since the vt library is compiled, you need to create the symbolic link of the folder \texttt{<morpheme-privat\_parent\_directory>/vt/build/bin} to \texttt{ASTEC/CommunFunctions/cpp}.
in a terminal:

 \texttt{`cd path/to/Package/ASTEC/CommunFunctions`}
 
 \texttt{ln -s <complete\_path\_of\_bin\_folder> cpp}
 
 
 \section{Troubleshooting}



