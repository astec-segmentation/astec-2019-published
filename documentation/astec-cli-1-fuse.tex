\section{\texttt{1-fuse.py}}
\label{sec:cli:fuse}

The fusion is made of the following steps.
\begin{enumerate}
\itemsep -1ex
\item \label{it:fusion:slit:line} Optionally, a slit line correction. Some Y lines may appear brighter in the acquisition and causes artifacts in the reconstructed (i.e. fused) image. By default, it is not done.

\item A change of resolution in the X and Y directions only (Z remains unchanged). It allows to decrease the data volume (and then the computational cost) if the new pixel size (set by \verb|target_resolution|) is larger than the acquisition one.

\item \label{it:fusion:crop:1} Optionally, a crop of the resampled acquisitions. It allows to decrease the volume of data, hence the computational cost. The crop is based on the analysis of a MIP view (in the Z direction) of  the volume, and thus is sensitive to hyper-intensities if any. By default, it is done.

\item Optionally, a mirroring of the 'right' image. It depends on the value of  \verb|raw_mirrors| variable.

\item \label{it:fusion:registration} Linear registration of the 3 last images on the first one (considered as the reference). The reference image is resampled again, to get an isotropic voxel (whose size is given by \verb|target_resolution|), i.e. the voxel size is the same along the 3 directions: X, Y, Z.

\item Linear combination of images, weighted by an ad-hoc function.

\item  \label{it:fusion:crop:2} Optionally, a crop of the fused image, still based on the analysis of a MIP view (in the Z direction). By default, it is done.
\end{enumerate}




\subsection{\texttt{1-fuse.py} options}

The following options are available:
\begin{description}
  \itemsep -1ex
\item[\texttt{-h}] prints a help message
\item[\texttt{-p \underline{file}}] indicates the parameter file to be parsed
\item[\texttt{-e \underline{path}}] indicates the
  \texttt{\underline{path}} to the directory where the
  \texttt{RAWDATA/} directory is located
\item[\texttt{-k}] allows to keep the temporary files
\item[\texttt{-f}] forces execution, even if (temporary) result files
  are already existing
\item[\texttt{-v}] increases verboseness (both at console and in the
  log file)
\item[\texttt{-nv}] no verboseness
\item[\texttt{-d}]  increases debug information (in the
  log file)
\item[\texttt{-nd}] no debug information
\end{description}


\subsection{Important parameters in the parameter file}
\label{sec:cli:fuse:important:parameters}

A simple parameter file for fusion is described in the tutorial
section \ref{sec:tutorial:fusion}. A more comprehensive parameter file
example is provided in the \texttt{parameter-file-examples/} directory.

Indicating the right values of the
acquisition parameters is crucial; these parameters are
\begin{itemize}
\itemsep -1ex
\item \texttt{raw\_ori} is a parameter describing the acquisition
  orientation of the acquisition of the second pair of images. Its value can be either \texttt{'left'} (orientation of
  $270 \degree$) or \texttt{'right'} (orientation of
  $90 \degree$).
\item \texttt{raw\_mirrors} is a parameter indicating whether the
  right camera images have to be mirrored or not. Its value is either
  \texttt{False} or \texttt{True}.
\item \texttt{raw\_resolution} is the voxel size (along the 3
    dimensions X, Y and Z) of the acquired images.
\item \texttt{target\_resolution} is the desired isotropic (the
    same along the 3 dimensions) voxel size for the result fusion
    images.
\item \texttt{begin} gives the index of the first time point to be
  processed.
\item \texttt{end} gives the index of the last time point to be processed.
\end{itemize}


When one may not be sure of the \texttt{raw\_ori} and
\texttt{raw\_mirrors} right values, it is advised to perform the
fusion on only one time point (by indicating the same index for both
\texttt{begin}  and \texttt{end}), with the four possibilities for the
variable couple (\texttt{raw\_ori}, \texttt{raw\_mirrors}), i.e.
(\texttt{'left'}, \texttt{False}),
(\texttt{'left'}, \texttt{True}),
(\texttt{'right'}, \texttt{False}), and
(\texttt{'right'}, \texttt{True}).
It comes to write four parameter files that differ only for the
parameters \texttt{raw\_ori}, \texttt{raw\_mirrors}, and
\texttt{EXP\_FUSE}  (to store the fusion result in different
directories, see section \ref{sec:cli:fuse:output:data}).
For these first experiments, it is also advised to set
\texttt{target\_resolution} to a large value, in order to speed up the
calculations.



\subsection{Input data}
\label{sec:cli:fuse:input:data}

Input data (acquired images from the MuViSPIM microscope) are assumed
to be organized in a separate \texttt{RAWDATA/} directory in the 
\texttt{/path/to/experiment/} directory as depicted below. 
\begin{itemize}
  \itemsep -1ex
\item \texttt{RAWDATA/LC/Stack000} contains the images acquired at the
  first angulation by the left camera.
\item \texttt{RAWDATA/LC/Stack001} contains the images acquired at the
  second angulation by the left camera.
\item \texttt{RAWDATA/RC/Stack000} contains the images acquired at the
  first angulation by the right camera.
\item \texttt{RAWDATA/RC/Stack001} contains the images acquired at the
  second angulation by the right camera.
\end{itemize}

\mbox{}
\dirtree{%
.1 /path/to/experiment/.
.2 RAWDATA/.
.3 LC/.
.4 Stack0000/.
.5 Time000xxx\_00.zip.
.5 {\ldots}.
.5 Time000xxx\_00.zip.
.4 Stack0001/.
.5 Time000xxx\_00.zip.
.5 {\ldots}.
.5 Time000xxx\_00.zip.
.3 RC/.
.4 Stack0000/.
.5 Time000xxx\_00.zip.
.5 {\ldots}.
.5 Time000xxx\_00.zip.
.4 Stack0001/.
.5 Time000xxx\_00.zip.
.5 {\ldots}.
.5 Time000xxx\_00.zip.
.2 \ldots.
}
\mbox{}

where \texttt{xxx} denotes a three digit number (e.g. $000$, $001$,
...) denoting the time point of each acquisition. The range of time
points to be fused are given by the variables \texttt{begin} and
\texttt{end}, while the path \texttt{/path/to/experiment/} has to be
assigned to the variable \texttt{PATH\_EMBRYO} 

Hence a parameter file containing
\begin{verbatim}
PATH_EMBRYO = /path/to/experiment/
begin = 0
end = 10
\end{verbatim}
indicates that time points in $[0,10]$ of the \texttt{RAWDATA/}
subdirectory of  \texttt{/path/to/experiment/} have to be fused.

\subsubsection{Input data directory names}

However, directories may be named differently. The variables
\texttt{DIR\_RAWDATA}, \texttt{DIR\_LEFTCAM\_STACKZERO},
\texttt{DIR\_RIGHTCAM\_STACKZERO}, \texttt{DIR\_LEFTCAM\_STACKONE},
and \texttt{DIR\_RIGHTCAM\_STACKONE} allow a finer control of the
directory names. The images acquired at the first angulation by the
left and the right cameras are searched in the directories
\begin{verbatim}
<PATH_EMBRYO>/<DIR_RAWDATA>/<DIR_LEFTCAM_STACKZERO>
<PATH_EMBRYO>/<DIR_RAWDATA>/<DIR_RIGHTCAM_STACKZERO>
\end{verbatim}
while the images acquired at the second angulation by the
left and the right cameras are searched in the directories
\begin{verbatim}
<PATH_EMBRYO>/<DIR_RAWDATA>/<DIR_LEFTCAM_STACKONE>
<PATH_EMBRYO>/<DIR_RAWDATA>/<DIR_RIGHTCAM_STACKONE>
\end{verbatim}
where \texttt{<XXX>} denotes the value of the variable \texttt{XXX}.
Then, to parse the following data architecture

\mbox{}
\dirtree{%
.1 /path/to/experiment/.
.2 my\_raw\_data/.
.3 LeftCamera/.
.4 FirstStack/.
.5 {\ldots}.
.4 SecondStack/.
.5 {\ldots}.
.3 RightCamera/.
.4 FirstStack/.
.5 {\ldots}.
.4 SecondStack/.
.5 {\ldots}.
.2 \ldots.
}
\mbox{}

one has to add the following lines in the parameter file
\begin{verbatim}
DIR_RAWDATA = 'my_raw_data'
DIR_LEFTCAM_STACKZERO = 'LeftCamera/FirstStack'
DIR_RIGHTCAM_STACKZERO = 'RightCamera/FirstStack'
DIR_LEFTCAM_STACKONE = 'LeftCamera/SecondStack'
DIR_RIGHTCAM_STACKONE = 'RightCamera/SecondStack'
\end{verbatim}

It has to be noted that, when the stacks of a given time point are in
different directories, image file names are tried to be guessed from
the directories parsing. It has to be pointed out that indexes have to
be encoded with a 3-digit integer with 0 padding (i.e. $000$, $001$,
\ldots) and that has to be the only variation in the file names
(within each directory).

\subsubsection{Input data image file names}

Images acquired from the left and the right cameras may be stored in
the same directory, but obviously with different names as in 

\mbox{}
\dirtree{%
.1 /path/to/experiment/.
.2 RAWDATA/.
.3 stack\_0\_channel\_0.
.4 Cam\_Left\_00xxx.zip.
.4  \ldots .
.4 Cam\_Right\_00xxx.zip.
.4 \ldots .
.3 stack\_1\_channel\_0.
.4 Cam\_Left\_00xxx.zip.
.4  \ldots .
.4 Cam\_Right\_00xxx.zip.
.4 \ldots .
}
\mbox{}

The parameter file has then to contain the following lines to indicate
the directory names.
\begin{verbatim}
DIR_LEFTCAM_STACKZERO = 'stack_0_channel_0'
DIR_RIGHTCAM_STACKZERO = 'stack_0_channel_0'
DIR_LEFTCAM_STACKONE = 'stack_1_channel_0'
DIR_RIGHTCAM_STACKONE = 'stack_1_channel_0'
\end{verbatim}

In addition, to distinguish the images acquired by the left camera to
those acquired by the right one, one has to give the image name
prefixes, i.e. the common part of the image file names before the
3-digit number that indicates the time point.
This is the purpose of the  variables
\verb|acquisition_leftcam_image_prefix| and 
\verb|acquisition_rightcam_image_prefix|.
The parameter file has then to contain the following lines not only to indicate
the directory names but also the image file name prefixes.

\begin{verbatim}
DIR_LEFTCAM_STACKZERO = 'stack_0_channel_0'
DIR_RIGHTCAM_STACKZERO = 'stack_0_channel_0'
DIR_LEFTCAM_STACKONE = 'stack_1_channel_0'
DIR_RIGHTCAM_STACKONE = 'stack_1_channel_0'
acquisition_leftcam_image_prefix = 'Cam_Left_00'
acquisition_rightcam_image_prefix = 'Cam_Right_00'
\end{verbatim}

\subsubsection{Multichannel acquisition}

In case of multichannel acquisition, the fusion is computed for the
first channel, and the computed parameters (e.g. transformations,
etc.) are also used for the other channels. 

For a second channel, 
the images acquired at the first angulation by the
left and the right cameras are searched in the directories
\begin{verbatim}
<PATH_EMBRYO>/<DIR_RAWDATA>/<DIR_LEFTCAM_STACKZERO_CHANNEL_2>
<PATH_EMBRYO>/<DIR_RAWDATA>/<DIR_RIGHTCAM_STACKZERO_CHANNEL_2>
\end{verbatim}
while the images acquired at the second angulation by the
left and the right cameras are searched in the directories
\begin{verbatim}
<PATH_EMBRYO>/<DIR_RAWDATA>/<DIR_LEFTCAM_STACKONE_CHANNEL_2>
<PATH_EMBRYO>/<DIR_RAWDATA>/<DIR_RIGHTCAM_STACKONE_CHANNEL_2>
\end{verbatim}

For a third channel, 
the images acquired at the first angulation by the
left and the right cameras are searched in the directories
\begin{verbatim}
<PATH_EMBRYO>/<DIR_RAWDATA>/<DIR_LEFTCAM_STACKZERO_CHANNEL_3>
<PATH_EMBRYO>/<DIR_RAWDATA>/<DIR_RIGHTCAM_STACKZERO_CHANNEL_3>
\end{verbatim}
while the images acquired at the second angulation by the
left and the right cameras are searched in the directories
\begin{verbatim}
<PATH_EMBRYO>/<DIR_RAWDATA>/<DIR_LEFTCAM_STACKONE_CHANNEL_3>
<PATH_EMBRYO>/<DIR_RAWDATA>/<DIR_RIGHTCAM_STACKONE_CHANNEL_3>
\end{verbatim}



\subsection{Output data}
\label{sec:cli:fuse:output:data}

The variable \texttt{target\_resolution} allows to set the desired isotropic (the
same along the 3 dimensions) voxel size for the result fusion
images.

\subsubsection{Output data directory names}

The resulting fused images are stored in sub-directories
\texttt{FUSE/FUSE\_<EXP\_FUSE>} under the \texttt{/path/to/experiment/} directory

\mbox{}
\dirtree{%
.1 /path/to/experiment/.
.2 RAWDATA/.
.3 \ldots.
.2 FUSE/.
.3 FUSE\_<EXP\_FUSE>/.
.4 \ldots.
}
\mbox{}

where \texttt{<EXP\_FUSE>} is the value of the variable \texttt{EXP\_FUSE} (its
default value is '\texttt{RELEASE}'). Hence, the line
\begin{verbatim}
EXP_FUSE = 'TEST'
\end{verbatim}
in the parameter file will create the directory
\texttt{FUSE/FUSE\_TEST/} in which the fused images are stored. For
instance, when testing for the values of the variable couple
(\texttt{raw\_ori}, \texttt{raw\_mirrors}), a first parameter file may
contain
\begin{verbatim}
raw_ori = 'left'
raw_mirrors = False
begin = 1
end = 1
EXP_FUSE=TEST-LEFT-FALSE
\end{verbatim}
a second parameter file may
contain
\begin{verbatim}
raw_ori = 'left'
raw_mirrors = True
begin = 1
end = 1
EXP_FUSE=TEST-LEFT-TRUE
\end{verbatim}
etc. The resulting fused images will then be in different directories

\mbox{}
\dirtree{%
.1 /path/to/experiment/.
.2 RAWDATA/.
.3 \ldots.
.2 FUSE/.
.3 FUSE\_TEST-LEFT-FALSE/.
.4 \ldots.
.3 FUSE\_TEST-LEFT-TRUE/.
.4 \ldots.
.3 \ldots.
}
\mbox{}

This will ease their visual inspection to decide which values of the variable couple
(\texttt{raw\_ori}, \texttt{raw\_mirrors}) to use for the fusion.

\subsubsection{Output data file names}

Fused image files are named after the variable \texttt{EN}:
\texttt{<EN>\_fuse\_t<xxx>.inr} where \texttt{<xxx>} is the time point
index encoded by a 3-digit integer (with 0 padding).


\subsubsection{Multichannel acquisition}

Variables \texttt{EXP\_FUSE\_CHANNEL\_2} and
\texttt{EXP\_FUSE\_CHANNEL\_3} allows to set the directory names for
the resulting fused images of the other channels.

\subsection{Fusion parameters}

% \subsubsection{Step \ref{it:fusion:slit:line}: slit line correction}

\subsubsection{Step \ref{it:fusion:crop:1}: raw data cropping}
\label{sec:cli:fuse:raw:data:cropping}

For computational cost purposes, raw data (images acquired by the MuViSPIM microscope) are cropped (only in X and Y dimensions) before co-registration. A threshold is computed with Otsu's method \cite{otsu:tsmc:1979} on the maximum intensity projection (MIP) image. The cropping parameters are computed to keep the above-threshold points in the MIP image, plus some extra margins. Hyper-intense areas may biased the threshold computation, hence the cropping.

To desactivate this cropping, the line
\begin{verbatim}
raw_crop = False
\end{verbatim}
has to be added in the parameter file.

\subsubsection{Step \ref{it:fusion:registration}: linear registration}
\label{sec:cli:fuse:linear:registration}

To decrease the computational cost, images are normalized and cast on one byte before registration. While it generally does not degrade the registration quality, it may induce troubles when hyper-intensities areas are present in the image. In such a case, the useful information may then be summarized in only a few intensity values.

Intensity normalization in registration can be deactivated by adding the following line in the parameter file
\begin{verbatim}
fusion_registration_normalization = False
\end{verbatim}

To verify whether a good quality registration can be conducted, the searched transformation type can be changed for a simpler one than affine. 
Adding the following line in the parameter file.
\begin{verbatim}
fusion_registration_transformation_type = translation
\end{verbatim}
will search for a translation which is supposed to be sufficient, according that only translations relates the 4 acquisitions of the MuViSPIM microscope (in a perfect setting). If the search for an affine transformation (the default behavior) failed (the fusion looks poor) while the search for a translation is successful (the fusion looks good), a two-steps registration may help to refine the found translation by a subsequent affine transformation as explained below.

Hyper-intensities areas may also bias the threshold calculation used for the automatic crop (step \ref{it:fusion:crop:1} of fusion). In such cases, the iterative registration method may find a local minimum that is not the desired one, because the relative positions of the two images to be co-registered are too far apart. To circumvent such a behavior, a two-steps registration can be done. It consists on a first pre-registration with a transformation with fewer degrees of freedom (i.e. a 3D translation). 

This pre-registration can be activated by adding the following line in the parameter file.
\begin{verbatim}
fusion_preregistration_compute_registration = True
\end{verbatim}
It may be also preferable to  deactivate the image normalization for both registration steps with
\begin{verbatim}
fusion_preregistration_normalization = False
fusion_registration_normalization = False
\end{verbatim}



\subsubsection{Step \ref{it:fusion:crop:2}: fused data cropping}
\label{sec:cli:fuse:fused:data:cropping}

To save disk storage, fused images are cropped at the end of the fusion stage. To desactivate this cropping, the line
\begin{verbatim}
fusion_crop = False
\end{verbatim}
has to be added in the parameter file.

\subsection{Troubleshooting}

\begin{itemize}
\item The fused images are obviously wrong.
  \begin{enumerate}
  \item Are the values of the variable couple (\texttt{raw\_ori}, \texttt{raw\_mirrors}) the right ones? Conduct experiments as suggested in section \ref{sec:cli:fuse:important:parameters}  (see also section \ref{sec:cli:fuse:output:data}) to get the right values.
  \item The registration may have failed.
    \begin{enumerate}
    \item Deactivate the 1-byte normalization (see section \ref{sec:cli:fuse:linear:registration}).
    \item Try to register with a simpler transformation type (i.e. translation) and/or with a two-steps registration (see section \ref{sec:cli:fuse:linear:registration}).
    \end{enumerate}
  \end{enumerate}
\item The imaged sample is cropped by the image border in the fused image.
  \begin{enumerate}
  \item Check whether the imaged sample was not already cropped in the raw data.
  \item The automated cropping may have failed. It is more likely to happen when cropping the raw data, so deactivate it (see section \ref{sec:cli:fuse:raw:data:cropping}). If it still happens, try to deactivate also the fused image cropping   (see section \ref{sec:cli:fuse:fused:data:cropping}).
  \end{enumerate}
\end{itemize}

\subsection{Parameter list}

Please also refer to the file
\texttt{parameter-file-examples/1-fuse-parameters.py}

\begin{itemize}
\itemsep -1ex
\item \texttt{DIR\_LEFTCAM\_STACKONE} see section \ref{sec:cli:fuse:input:data}
\item \texttt{DIR\_LEFTCAM\_STACKONE\_CHANNEL\_2} see section \ref{sec:cli:fuse:input:data}
\item \texttt{DIR\_LEFTCAM\_STACKONE\_CHANNEL\_3} see section \ref{sec:cli:fuse:input:data}
\item \texttt{DIR\_LEFTCAM\_STACKZERO} see section \ref{sec:cli:fuse:input:data}
\item \texttt{DIR\_LEFTCAM\_STACKZERO\_CHANNEL\_2} see section \ref{sec:cli:fuse:input:data}
\item \texttt{DIR\_LEFTCAM\_STACKZERO\_CHANNEL\_3} see section \ref{sec:cli:fuse:input:data}
\item \texttt{DIR\_RAWDATA} see section \ref{sec:cli:fuse:input:data}
\item \texttt{DIR\_RAWDATA\_CHANNEL\_2} see section \ref{sec:cli:fuse:input:data}
\item \texttt{DIR\_RAWDATA\_CHANNEL\_3} see section \ref{sec:cli:fuse:input:data}
\item \texttt{DIR\_RIGHTCAM\_STACKONE} see section \ref{sec:cli:fuse:input:data}
\item \texttt{DIR\_RIGHTCAM\_STACKONE\_CHANNEL\_2} see section \ref{sec:cli:fuse:input:data}
\item \texttt{DIR\_RIGHTCAM\_STACKONE\_CHANNEL\_3} see section \ref{sec:cli:fuse:input:data}
\item \texttt{DIR\_RIGHTCAM\_STACKZERO} see section \ref{sec:cli:fuse:input:data}
\item \texttt{DIR\_RIGHTCAM\_STACKZERO\_CHANNEL\_2} see section \ref{sec:cli:fuse:input:data}
\item \texttt{DIR\_RIGHTCAM\_STACKZERO\_CHANNEL\_3} see section \ref{sec:cli:fuse:input:data}
\item \texttt{EN} see section \ref{sec:cli:fuse:output:data}
\item \texttt{EXP\_FUSE} see section \ref{sec:cli:fuse:output:data}
\item \texttt{EXP\_FUSE\_CHANNEL\_2} see section \ref{sec:cli:fuse:output:data}
\item \texttt{EXP\_FUSE\_CHANNEL\_3} see section \ref{sec:cli:fuse:output:data}
\item \texttt{PATH\_EMBRYO} see section \ref{sec:cli:fuse:input:data}
\item \texttt{RESULT\_IMAGE\_SUFFIX\_FUSE}
\item \texttt{acquisition\_leftcam\_image\_prefix}  see section \ref{sec:cli:fuse:input:data}
\item \texttt{acquisition\_mirrors} same as \texttt{raw\_mirrors}
\item \texttt{acquisition\_orientation} same as \texttt{raw\_ori}
\item \texttt{acquisition\_resolution} same as \texttt{raw\_resolution}
\item \texttt{acquisition\_rightcam\_image\_prefix}  see section \ref{sec:cli:fuse:input:data}
\item \texttt{acquisition\_slit\_line\_correction}
\item \texttt{begin} see section \ref{sec:cli:fuse:important:parameters}
\item \texttt{default\_image\_suffix}
\item \texttt{delta}
\item \texttt{end} see section \ref{sec:cli:fuse:important:parameters}
\item \texttt{fusion\_crop} see section \ref{sec:cli:fuse:fused:data:cropping}
\item \texttt{fusion\_margin\_x\_0}
\item \texttt{fusion\_margin\_x\_1}
\item \texttt{fusion\_margin\_y\_0}
\item \texttt{fusion\_margin\_y\_1}
\item \texttt{fusion\_preregistration\_compute\_registration} see section \ref{sec:cli:fuse:linear:registration}
\item \texttt{fusion\_preregistration\_lts\_fraction}
\item \texttt{fusion\_preregistration\_normalization} see section \ref{sec:cli:fuse:linear:registration}
\item \texttt{fusion\_preregistration\_pyramid\_highest\_level}
\item \texttt{fusion\_preregistration\_pyramid\_lowest\_level}
\item \texttt{fusion\_preregistration\_transformation\_estimation\_type}
\item \texttt{fusion\_preregistration\_transformation\_type}
\item \texttt{fusion\_registration\_compute\_registration}
\item \texttt{fusion\_registration\_lts\_fraction}
\item \texttt{fusion\_registration\_normalization} see section \ref{sec:cli:fuse:linear:registration}
\item \texttt{fusion\_registration\_pyramid\_highest\_level}
\item \texttt{fusion\_registration\_pyramid\_lowest\_level}
\item \texttt{fusion\_registration\_transformation\_estimation\_type}
\item \texttt{fusion\_registration\_transformation\_type} see section \ref{sec:cli:fuse:linear:registration}
\item \texttt{raw\_crop} see section \ref{sec:cli:fuse:raw:data:cropping}
\item \texttt{raw\_delay}
\item \texttt{raw\_margin\_x\_0}
\item \texttt{raw\_margin\_x\_1}
\item \texttt{raw\_margin\_y\_0}
\item \texttt{raw\_margin\_y\_1}
\item \texttt{raw\_mirrors} see section \ref{sec:cli:fuse:important:parameters}
\item \texttt{raw\_ori} see section \ref{sec:cli:fuse:important:parameters}
\item \texttt{raw\_resolution} see section \ref{sec:cli:fuse:important:parameters}
\item \texttt{result\_image\_suffix}
\item \texttt{target\_resolution} see section \ref{sec:cli:fuse:output:data}
\end{itemize}
