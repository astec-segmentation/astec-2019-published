\chapter{Tutorial}
\label{chap:tutorial}

\section*{Before starting}

It is advised to add to your \texttt{PATH} environment variable 
the paths to both the python and the C executable commands (the latter
is important in case of non-standard installation). So, Astec commands can be launched without specifying the complete path to the command.

It can be done in a terminal (and will be valid only for this terminal)
\begin{code}{0.8}
  \$ export PATH=\${PATH}:/path/to/astec/src \\
  \$ export PATH=\${PATH}:/path/to/astec/src/ASTEC/CommunFunctions/cpp/vt/build/bin
\end{code}
or by modifying a setup file (e.g. \texttt{bashrc}, \texttt{.profile}, \ldots).





\section{Tutorial data}

The directory \texttt{/path/to/astec/tutorial/tuto-astec1/}, also denoted by
\texttt{path/to/experiment/} or \texttt{<EXPERIMENT>}, contains
the \texttt{RAWDATA/} and \texttt{parameters/} sub-directories and a
\texttt{README} file

\mbox{}
\dirtree{%
.1 path/to/tuto-astec1/.
.2 RAWDATA/.
.2 README.
.2 parameters/.
}
\mbox{}

The \texttt{RAWDATA/} contains 21 time points (indexed from 0 to 20)
of subsampled (for file size
consideration) raw data from a 3D+t movie acquired by a MuViSPIM microscope.

\mbox{}
\dirtree{%
.1 RAWDATA/.
.2 LC/.
.3 Stack0000/.
.4 Time000000\_00.mha.gz.
.4 {\ldots}.
.4 Time000020\_00.mha.gz.
.3 Stack0001/.
.4 Time000000\_00.mha.gz.
.4 {\ldots}.
.4 Time000020\_00.mha.gz.
.2 RC/.
.3 Stack0000/.
.4 Time000000\_00.mha.gz.
.4 {\ldots}.
.4 Time000020\_00.mha.gz.
.3 Stack0001/.
.4 Time000000\_00.mha.gz.
.4 {\ldots}.
.4 Time000020\_00.mha.gz.
}
\mbox{}

where \texttt{LC/} and \texttt{LC/} stand respectively for the left
and the right cameras.





\section{Fusion}
\label{sec:tutorial:fusion}

We assume that we are located in the directory
\texttt{/path/to/astec/tutorial/tuto-astec1/}. Running the fusion is
done with
\begin{code}{0.8}
  \$ 1-fuse.py -p parameters/1-fuse-tutorial-parameters.py
\end{code}
\texttt{1-fuse-tutorial-parameters.py} being the dedicated parameter file  (figure \ref{fig:tutorial:parameter:fusion}).

\begin{figure}
\begin{framed}
\begin{verbatim}
     1	PATH_EMBRYO = '.'
     2	
     3	EN = '2019-Tutorial100'
     4	
     5	begin = 0
     6	end = 20
     7	
     8	acquisition_orientation = 'right'
     9	acquisition_mirrors = False
    10	acquisition_resolution = (1., 1., 1.)
    11	
    12	target_resolution = 1.0
\end{verbatim}
\end{framed}
\caption{\label{fig:tutorial:parameter:fusion} Tutorial parameter file
  for the fusion step (lines are numbered).}
\end{figure}

\begin{itemize}
  \itemsep -1ex
  \item The variable \texttt{PATH\_EMBRYO} is the path to the directory where
    the directory \texttt{RAWDATA/} is located. It can be either relative (as in the
    above example) or
    global (it could have been \texttt{/path/to/astec/tutorial/tuto-astec1/}).
  \item The variable \texttt{EN} is the prefix after which the result fusion images
    will be named. 
  \item The variables \texttt{begin} and \texttt{end} set respectively the
    first and the last index of the input time points to be processed.
  \item The variables \texttt{acquisition\_orientation} and \texttt{acquisition\_mirrors} are parameters
    describing the acquisition geometry.
  \item The variable \texttt{acquisition\_resolution} is the voxel size (along the 3
    dimensions X, Y and Z).
  \item The variable \texttt{target\_resolution} is the desired isotropic (the
    same along the 3 dimensions) voxel size for the result fusion images.
\end{itemize}

After processing, a \texttt{FUSE/} directory has been created

\mbox{}
\dirtree{%
.1 path/to/tuto-astec1/.
.2 FUSE/.
.2 RAWDATA/.
.2 README.
.2 parameters/.
}
\mbox{}

The \texttt{FUSE/} directory contains 
\mbox{}
\dirtree{%
.1 FUSE/.
.2 FUSE\_RELEASE/.
.3 2019-Tutorial100\_fuse\_t000.inr.
.3 {\ldots}.
.3 2019-Tutorial100\_fuse\_t020.inr.
.3 LOGS/.
}
\mbox{}

The fused images are named after \texttt{<EN>\_fuse\t<XXX>.inr} (where \texttt{<XXX>}
denotes the value of the variable \texttt{XXX}) and indexed from
\texttt{<begin>} to \texttt{<end>} (as the input data).

The directory \texttt{LOGS/} contains
a copy of the parameter file (stamped with date and hour) as well as a
log file (also stamped with date and hour) reporting information about
the processing.



\section{Sequence intra-registration (or drift compensation) [1]}
\label{sec:tutorial:intra:registration:fuse}

We assume that we are located in the directory
\texttt{/path/to/astec/tutorial/tuto-astec1/}. Running the sequence intra-registration is
done with
\begin{code}{0.8}
  \$ 1.5-intraregistration.py -p parameters/1.5-intraregistration-tutorial-parameters-fuse.py 
\end{code}
\texttt{1.5-intraregistration-tutorial-parameters-fuse.py} being the
dedicated parameter file  (figure \ref{fig:tutorial:parameter:intra:registration:fuse}).

\begin{figure}
\begin{framed}
\begin{verbatim}
     1	PATH_EMBRYO = '.'
     2	
     3	EN = '2019-Tutorial100'
     4	
     5	begin = 0
     6	end = 20
\end{verbatim}
\end{framed}
\caption{\label{fig:tutorial:parameter:intra:registration:fuse} Tutorial
  parameter file for the sequence intra-registration step.}
\end{figure}

\begin{itemize}
  \itemsep -1ex
  \item The variable \texttt{PATH\_EMBRYO} is the path to the directory where
    the directory \texttt{FUSE/} is located. It can be either relative (as in the
    above example) or
    global (it could have been \texttt{/path/to/astec/tutorial/tuto-astec1/}).
  \item The variable \texttt{EN} is the prefix after which the  images
    are named. 
  \item The variables \texttt{begin} and \texttt{end} set respectively the
    first and the last index of the input time points to be processed.
\end{itemize}

After processing, a \texttt{INTRAREG/} directory has been created

\mbox{}
\dirtree{%
.1 path/to/tuto-astec1/.
.2 FUSE/.
.2 INTRAREG/.
.2 RAWDATA/.
.2 README.
.2 parameters/.
}
\mbox{}

The \texttt{INTRAREG/} directory contains

\mbox{}
\dirtree{%
.1 INTRAREG/.
.2 INTRAREG\_RELEASE/.
.3 CO-TRSFS/.
.4 2019-Tutorial100\_intrareg\_flo000\_ref001.trsf.
.4 \ldots.
.4 2019-Tutorial100\_intrareg\_flo019\_ref020.trsf.
.3 FUSE/.
.4 2019-Tutorial100\_intrareg\_fuse\_t000.inr.
.4 \ldots.
.4 2019-Tutorial100\_intrareg\_fuse\_t020.inr.
.3 LOGS/.
.3 MOVIES/.
.4 2019-Tutorial100\_intrareg\_fuse\_t0-20\_xy205.inr.
.3 TRSFS\_t0-20/.
.4 2019-Tutorial100\_intrareg\_t000.trsf.
.4 \ldots.
.4 2019-Tutorial100\_intrareg\_t020.trsf.
.4 template\_t0-20.inr.
}
\mbox{}

\begin{itemize}
\itemsep -1ex
\item The directory \texttt{CO-TRSF/} contains the co-registration
  transformations.
\item The directory \texttt{FUSE/} contains the resampled fused images
  in the same geometry (images have the same dimensions along X, Y and
  Z), with drift compensation (the eventual motion of the sample under the
  microscope has been compensated). 
\item The directory \texttt{MOVIES/} contains a 3D (which is a 2D+t)
  image, here
  \texttt{2019-Tutorial100\_intrareg\_fuse\_t0-20\_xy205.inr}, which
  the \#205 XY-section of the resampled fused images for all the time
  points.
\item The directory \texttt{TRSFS/}  contains the transformation of
  every fused image towards the reference one as well as the template
  image (an image large enough to including each fused images after
  resampling).

  The template image \texttt{template\_t0-20.inr} is of size $422
  \times 365 \times 410$ with a voxel size of 0.6 (the voxel size can
  be set by the variable \texttt{intra\_registration\_resolution})
\end{itemize}





\section{Segmentation of the first time point}


We assume that we are located in the directory
\texttt{/path/to/astec/tutorial/tuto-astec1/}. Segmenting the first
time point is
done with
\begin{code}{0.8}
  \$ 2-mars.py -p parameters/2-mars-tutorial-parameters.py 
\end{code}
\texttt{2-mars-tutorial-parameters.py} being the
dedicated parameter file  (figure \ref{fig:tutorial:parameter:mars}).

\begin{figure}
\begin{framed}
\begin{verbatim}
     1	PATH_EMBRYO = '.'
     2	
     3	EN = '2019-Tutorial100'
     4	
     5	begin = 0
\end{verbatim}
\end{framed}
\caption{\label{fig:tutorial:parameter:mars} Tutorial
  parameter file for the segmentation of the first time point.}
\end{figure}


\begin{itemize}
  \itemsep -1ex
  \item The variable \texttt{PATH\_EMBRYO} is the path to the directory where
    the directory \texttt{FUSE/} is located. It can be either relative (as in the
    above example) or
    global (it could have been \texttt{/path/to/astec/tutorial/tuto-astec1/}).
  \item The variable \texttt{EN} is the prefix after which the  images
    are named. 
  \item The variable \texttt{begin} sets  the
    index of the first input time point (to be processed).
\end{itemize}

After processing, a \texttt{SEG/} directory has been created

\mbox{}
\dirtree{%
.1 path/to/tuto-astec1/.
.2 FUSE/.
.2 INTRAREG/.
.2 RAWDATA/.
.2 README.
.2 SEG/.
.2 parameters/.
}
\mbox{}

The \texttt{SEG/} directory contains

\mbox{}
\dirtree{%
.1 SEG/.
.2 SEG\_RELEASE/.
.3 2019-Tutorial100\_mars\_t000.inr.
.3 LOGS/.
.3 RECONSTRUCTION/.
}
\mbox{}

\texttt{2019-Tutorial100\_mars\_t000.inr} is the segmented first
  time point of the sequence.



\section{Correction of  the first time point segmentation}
\label{sec:tutorial:manual:correction}


We assume that we are located in the directory
\texttt{/path/to/astec/tutorial/tuto-astec1/}. Segmenting the first
time point is
done with
\begin{code}{0.8}
  \$ 3-manualcorrection.py -p parameters/3-manualcorrection-tutorial-parameters.py 
\end{code}
\texttt{3-manualcorrection-tutorial-parameters.py} being the
dedicated parameter file  (figure \ref{fig:tutorial:parameter:manual:correction}).

\begin{figure}
\begin{framed}
\begin{verbatim}
     1	PATH_EMBRYO = '.'
     2	
     3	EN = '2019-Tutorial100'
     4	
     5	begin = 0
     6	
     7	mancor_mapping_file='parameters/3-manualcorrection-tutorial.txt'
\end{verbatim}
\end{framed}
\caption{\label{fig:tutorial:parameter:manual:correction} Tutorial
  parameter file for the segmentation correction of the first time
  point. See figure
  \protect\ref{fig:tutorial:parameter:manual:correction:file} for the
  \texttt{<mancor\_mapping\_file>} file.}
\end{figure}

\begin{figure}
\begin{framed}
\begin{verbatim}
10 6
20 13
9 4
26 11
21 11
27 15
32 23
39 31
35 20
38 43
46 45
52 42
58 62
63 60
78 67
74 66
68 66
83 75
82 77
\end{verbatim}
\end{framed}
\caption{\label{fig:tutorial:parameter:manual:correction:file}
  The segmentation correction file
  \texttt{3-manualcorrection-tutorial.txt} for the first time point. The first
  number is the line index (lines are numbered).}
\end{figure}

\begin{itemize}
  \itemsep -1ex
  \item The variable \texttt{PATH\_EMBRYO} is the path to the directory where
    the directory \texttt{SEG/} is located. It can be either relative (as in the
    above example) or
    global (it could have been \texttt{/path/to/astec/tutorial/tuto-astec1/}).
  \item The variable \texttt{EN} is the prefix after which the  images
    are named. 
  \item The variable \texttt{begin} set  the
    index of the first input time point (to be processed).
  \item The variable \texttt{mancor\_mapping\_file} gives the file
    name containing the correction to be applied.
\end{itemize}

After processing, the \texttt{SEG/} directory contains

\mbox{}
\dirtree{%
.1 SEG/.
.2 SEG\_RELEASE/.
.3 2019-Tutorial100\_mars\_t000.inr.
.3 2019-Tutorial100\_seg\_t000.inr.
.3 LOGS/.
.3 RECONSTRUCTION/.
}
\mbox{}

\texttt{2019-Tutorial100\_seg\_t000.inr} is the corrected version of
the segmentation obtained at the previous step.



\section{Segmentation propagation}
\label{sec:tutorial:segmentation:propagation}

We assume that we are located in the directory
\texttt{/path/to/astec/tutorial/tuto-astec1/}. Correcting the first
time point is
done with
\begin{code}{0.8}
  \$ 4-astec.py -p parameters/4-astec-tutorial-parameters.py  
\end{code}
\texttt{4-astec-tutorial-parameters.py} being the
dedicated parameter file  (figure \ref{fig:tutorial:parameter:astec}).

\begin{figure}
\begin{framed}
\begin{verbatim}
     1	PATH_EMBRYO = '.'
     2	
     3	EN = '2019-Tutorial100'
     4	
     5	begin = 0
     6	end = 20
     7	delta = 1
     8	raw_delay = 0
     9	
    10	## General parameters for segmentation propagation
    11	
    12	astec_sigma1 = 0.6  		
    13	astec_sigma2 = 0.15 		
    14	astec_h_min_min = 4
    15	astec_h_min_max = 18   		
    16	
    17	## Glace Parameters (if astec_membrane_reconstruction_method is set to 1 or 2):
    18	## membrane_renforcement
    19	
    20	astec_sigma_membrane = 0.9
    21	astec_sensitivity = 0.99  
    22	astec_manual = False     	
    23	astec_manual_sigma = 15   
    24	astec_hard_thresholding = False 
    25	astec_hard_threshold = 1.0      
    26	
    27	## Tensor voting framework
    28	
    29	astec_sigma_TV = 3.6    
    30	astec_sigma_LF = 0.9    
    31	astec_sample = 0.2      
    32	
    33	## Default parameters (for classical use, default values should not be changed)
    34	
    35	astec_RadiusOpening = 20 		
    36	astec_Thau = 25 				
    37	astec_MinVolume = 1000 		
    38	astec_VolumeRatioBigger = 0.5 
    39	astec_VolumeRatioSmaller = 0.1
    40	astec_MorphosnakeIterations = 10 
    41	astec_NIterations = 200 		
    42	astec_DeltaVoxels = 10**3  	
    43	astec_nb_proc = 10
\end{verbatim}
\end{framed}
\caption{\label{fig:tutorial:parameter:astec} Tutorial
  parameter file for the segmentation propagation.}
\end{figure}

After processing, the \texttt{SEG/} directory contains

\mbox{}
\dirtree{%
.1 SEG/.
.2 SEG\_RELEASE/.
.3 2019-Tutorial100\_mars\_t000.inr.
.3 2019-Tutorial100\_seg\_lineage.test.
.3 2019-Tutorial100\_seg\_t000.inr.
.3 2019-Tutorial100\_seg\_t001.inr.
.3 \ldots.
.3 2019-Tutorial100\_seg\_t020.inr.
.3 4-astec-tutorial-parameters.py.
.3 LOGS/.
.3 RECONSTRUCTION/.
}
\mbox{}

while a \texttt{2019-Tutorial100\_seg\_lineage.pkl} file has been
created in the \texttt{path/to/tuto-astec1/} directory.

\mbox{}
\dirtree{%
.1 path/to/tuto-astec1/.
.2 2019-Tutorial100\_seg\_lineage.pkl.
.2 FUSE/.
.2 INTRAREG/.
.2 RAWDATA/.
.2 README.
.2 SEG/.
.2 parameters/.
}
\mbox{}

\texttt{2019-Tutorial100\_seg\_lineage.pkl} is a pickle python file
containing a dictionary (in the python sense). It can be read by
\begin{code}{0.8}
  \$ python \\
  \ldots \\
  >>> import cPickle as pkl \\
  >>> f=open('2019-Tutorial100\_seg\_lineage.pkl') \\
  >>> d=pkl.load(f) \\
  >>> f.close()  \\
  >>> d.keys() \\{}
  ['h\_mins\_information', 'lin\_tree', 'volumes\_information', 'sigmas\_information']
\end{code}
In this pickle file, cells have an unique identifier $i * 1000 + c$, which is made of
both the image index $i$ and the cell identifier $c$ within a segmentation
image (recall that, within an image, cells are numbered from 2, 1 being the background
label).



\section{Sequence intra-registration (or drift compensation) [2]}
\label{sec:tutorial:intra:registration:seg}

We assume that we are located in the directory
\texttt{/path/to/astec/tutorial/tuto-astec1/}. Running the sequence intra-registration is
done with
\begin{code}{0.8}
  \$ 1.5-intraregistration.py -p parameters/1.5-intraregistration-tutorial-parameters-seg.py 
\end{code}
\texttt{1.5-intraregistration-tutorial-parameters-seg.py} being the
dedicated parameter file  (figure \ref{fig:tutorial:parameter:intra:registration:seg}).

\begin{figure}
\begin{framed}
\begin{verbatim}
     1	PATH_EMBRYO = '.'
     2	
     3	EN = '2019-Tutorial100'
     4	
     5	begin = 0
     6	end = 20
     7	
     8	EXP_INTRAREG = 'SEG'
     9	
    10	intra_registration_template_type = "SEGMENTATION"
    11	intra_registration_template_threshold = 2
    12	intra_registration_margin = 20
    13	
    14	intra_registration_resample_segmentation_images = True
    15	intra_registration_movie_segmentation_images = True
\end{verbatim}
\end{framed}
\caption{\label{fig:tutorial:parameter:intra:registration:seg} Tutorial
  parameter file for the sequence intra-registration step,
  segmentation images being used to build the template.}
\end{figure}

\begin{itemize}
  \itemsep -1ex
  \item The variable \texttt{PATH\_EMBRYO} is the path to the directory where
    the directory \texttt{FUSE/} is located. It can be either relative (as in the
    above example) or
    global (it could have been \texttt{/path/to/astec/tutorial/tuto-astec1/}).
  \item The variable \texttt{EN} is the prefix after which the  images
    are named. 
  \item The variables \texttt{begin} and \texttt{end} set respectively the
    first and the last index of the input time points to be processed.
  \item  the variable \texttt{EXP\_INTRAREG} set the suffix of the
    sub-directory of the \texttt{INTRAREG/} directory to be created.
  \item  the variable \texttt{intra\_registration\_template\_type} set
    the images to be used to build the template. Here, since it is
    equal to \texttt{"SEGMENTATION''}, they are the
    segmentation images obtained at the previous step.

    The variable \texttt{intra\_registration\_template\_threshold} set
    a threshold to be applied to the template images to define the
    information to be kept: we want all the points with a value equal
    or greater than 2 to be contained in the template after
    resampling. Since cells are labeled from 2 and above, the template
    is designed to contain all labeled cells after resampling, so it
    is built as small as possible.

    The variable \texttt{intra\_registration\_margin} allows to add
    margins (in the 3 dimensions) to the built template.
    
  \item The variable
    \texttt{intra\_registration\_resample\_segmentation\_images}
    indicates whether the segmentation images are to be resampled in
    the template geometry. 
  \item The variable
    \texttt{intra\_registration\_movie\_segmentation\_images}
    indicates whether 2D+t movies have to be built from the resampled
    segmentation images.
\end{itemize}

After processing, a \texttt{INTRAREG/INTRAREG\_SEG/} directory has
been created and the \texttt{INTRAREG/} directory now contains

\mbox{}
\dirtree{%
.1 INTRAREG/.
.2 INTRAREG\_RELEASE/.
.3 \ldots.
.2 INTRAREG\_SEG/.
.3 CO-TRSFS/.
.4 2019-Tutorial100\_intrareg\_flo000\_ref001.trsf.
.4 \ldots.
.4 2019-Tutorial100\_intrareg\_flo019\_ref020.trsf.
.3 FUSE/.
.4 2019-Tutorial100\_intrareg\_fuse\_t000.inr.
.4 \ldots.
.4 2019-Tutorial100\_intrareg\_fuse\_t020.inr.
.3 LOGS/.
.3 MOVIES/.
.4 2019-Tutorial100\_intrareg\_fuse\_t0-20\_xy174.inr.
.4 2019-Tutorial100\_intrareg\_seg\_t0-20\_xy174.inr.
.3 SEG/.
.4 2019-Tutorial100\_intrareg\_seg\_t000.inr.
.4 \ldots.
.4 2019-Tutorial100\_intrareg\_seg\_t020.inr.
.3 TRSFS\_t0-20/.
.4 2019-Tutorial100\_intrareg\_t000.trsf.
.4 \ldots.
.4 2019-Tutorial100\_intrareg\_t020.trsf.
.4 template\_t0-20.inr.
}
\mbox{}

In addition to directories already described in section
\ref{sec:tutorial:intra:registration:fuse}, the \texttt{INTRAREG\_SEG/} directory contains
\begin{itemize}
\itemsep -1ex
\item The directory \texttt{SEG/} contains the resampled segmentation images
  in the same geometry (images have the same dimensions along X, Y and
  Z), with drift compensation (the eventual motion of the sample under the
  microscope has been compensated). 
\item In addition to a 2D+t movie made from the resampled fusion
  images, the directory \texttt{MOVIES/} contains a 2D+t movie made from the resampled segmentation
  images.
\item The template image \texttt{template\_t0-20.inr} in the directory
  \texttt{TRSFS/} 
  is now of size $323 \times 265 \times 348$ with a voxel size of 0.6,
  which is smaller than the one computed in section
  \ref{sec:tutorial:intra:registration:fuse}, even with the added
  margins.
  
  Note that all resampled images (in both the \texttt{FUSE/} and the
  \texttt{SEG/} directories have the same geometry than the template
  image. 
\end{itemize}


\section{Sequence properties computation [1]}
\label{sec:tutorial:properties:seg}

We assume that we are located in the directory
\texttt{/path/to/astec/tutorial/tuto-astec1/}.
Computing cell properties as well as lineage assumes that
 segmentation or post-corrected segmentation (see section
\ref{sec:tutorial:properties:post}) images have been
co-registered (see sections \ref{sec:tutorial:intra:registration:seg}
and \ref{sec:tutorial:intra:registration:post}). 
Extracting the sequence properties from the co-registered segmentation
images is
done with
\begin{code}{0.8}
  \$ X-embryoproperties.py -p parameters/X-embryoproperties-tutorial-parameters-seg.py
\end{code}
\texttt{X-embryoproperties-tutorial-parameters-seg.py} being the
dedicated parameter file  (figure \ref{fig:tutorial:parameter:properties:seg}).

\begin{figure}
\begin{framed}
\begin{verbatim}
     1	PATH_EMBRYO = '.'
     2	
     3	EN = '2019-Tutorial100'
     4	
     5	begin = 0
     6	end = 20
     7	
     8	EXP_INTRAREG = 'SEG'
\end{verbatim}
\end{framed}
\caption{\label{fig:tutorial:parameter:properties:seg} Tutorial
  parameter file for the sequence properties from the co-registered
  segmentation images.}
\end{figure}

\begin{itemize}
  \itemsep -1ex
  \item The variable \texttt{PATH\_EMBRYO} is the path to the directory where
    the directory \texttt{FUSE/} is located. It can be either relative (as in the
    above example) or
    global (it could have been \texttt{/path/to/astec/tutorial/tuto-astec1/}).
  \item The variable \texttt{EN} is the prefix after which the  images
    are named. 
  \item The variables \texttt{begin} and \texttt{end} set respectively the
    first and the last index of the input time points to be processed.
  \item  the variable \texttt{EXP\_INTRAREG} set the suffix of the
    sub-directory of the \texttt{INTRAREG/} directory where to search
    post-corrected segmentation or segmentation images.

    Since the directory \texttt{INTRAREG/INTRAREG\_SEG/} does only
    contains the co-registered segmentation images (in the
    \texttt{SEG/} sub-directory), properties will be computed from
    these images.
\end{itemize}

After processing, some files appears in the \texttt{INTRAREG/INTRAREG\_SEG/SEG/} sub-directory 

\mbox{}
\dirtree{%
.1 INTRAREG/.
.2 INTRAREG\_RELEASE/.
.3 \ldots.
.2 INTRAREG\_SEG/.
.3 CO-TRSFS/.
.4 \ldots.
.3 FUSE/.
.4 \ldots.
.3 LOGS/.
.3 MOVIES/.
.4 \ldots.
.3 SEG/.
.4 2019-Tutorial100\_intrareg\_seg\_lineage.pkl.
.4 2019-Tutorial100\_intrareg\_seg\_lineage.txt.
.4 2019-Tutorial100\_intrareg\_seg\_lineage.xml.
.4 2019-Tutorial100\_intrareg\_seg\_t000.inr.
.4 \ldots.
.4 2019-Tutorial100\_intrareg\_seg\_t020.inr.
.3 TRSFS\_t0-20/.
.4 \ldots.
}
\mbox{}


\texttt{2019-Tutorial100\_intrareg\_seg\_lineage.pkl} is a pickle python file
containing a dictionary (in the python sense). It can be read by
\begin{code}{0.8}
  \$ python \\
  \ldots \\
  >>> import cPickle as pkl \\
  >>> f=open('2019-Tutorial100\_intrareg\_seg\_lineage.pkl') \\
  >>> d=pkl.load(f) \\
  >>> f.close()  \\
  >>> d.keys() \\{}
  ['all\_cells', 'cell\_barycenter', 'cell\_contact\_surface', 'cell\_principal\_vectors', 'cell\_principal\_values', 'cell\_volume', 'cell\_compactness', 'cell\_surface', 'cell\_lineage']
\end{code}
In this pickle file (as in the one computed at section
\ref{sec:tutorial:segmentation:propagation}), cells have an unique
identifier $i * 1000 + c$, which is made of 
both the image index $i$ and the cell identifier $c$ within a segmentation
image (recall that cells are numbered from 2, 1 being the background
label).

\texttt{2019-Tutorial100\_intrareg\_seg\_lineage.xml} contains the
same information than the pickle file, but in xml format (see figure
\ref{fig:tutorial:seg:properties:xml}).

\begin{figure}
\begin{framed}
\begin{verbatim}
<data>
  <cell_volume>
    ...
  </cell_volume>
  <cell_surface>
     ...
  </cell_surface>
  <cell_compactness>
     ...
  </cell_compactness>
  <cell_barycenter>
     ...
  </cell_barycenter>
  <cell_principal_values>
     ...
  </cell_principal_values>
  <cell_principal_vectors>
     ...
  </cell_principal_vectors>
  <cell_contact_surface>
     ...
  </cell_contact_surface>
  <all_cells>[2, 3, 4, 5, 6, 7, 8, 11, 12, 13, 
     ...
     200097, 200099, 200100, 200101, 200102]</all_cells>
  <cell_lineage>
     ...
  </cell_lineage>
</data>
\end{verbatim}
\end{framed}
\caption{\label{fig:tutorial:seg:properties:xml} XML output properties
  file from the co-registered segmentation image.}
\end{figure}

\texttt{2019-Tutorial100\_intrareg\_seg\_lineage.tst} contains some
\textit{diagnosis} information (smallest and largest cells, weird
lineages, etc.). 

\section{Segmentation post-correction}

We assume that we are located in the directory
\texttt{/path/to/astec/tutorial/tuto-astec1/}. Segmentation post-correction is
done with
\begin{code}{0.8}
  \$ 5-postcorrection.py -p parameters/5-postcorrection-tutorial-parameters.py 
\end{code}
\texttt{5-postcorrection-tutorial-parameters.py } being the
dedicated parameter file  (figure
\ref{fig:tutorial:parameter:post:correction}).


\begin{figure}
\begin{framed}
\begin{verbatim}
     1	PATH_EMBRYO = '.'
     2	
     3	EN = '2019-Tutorial100'
     4	
     5	begin = 0
     6	end = 20
     7	delta = 1
     8	raw_delay = 0
     9	
    10	postcor_Volume_Threshold=10000 	
    11	postcor_Soon=True 				
\end{verbatim}
\end{framed}
\caption{\label{fig:tutorial:parameter:post:correction} Tutorial
  parameter file for the segmentation post-correction.}
\end{figure}

After processing, a \texttt{POST/} directory has been created

\mbox{}
\dirtree{%
.1 path/to/tuto-astec1/.
.2 2019-Tutorial100\_seg\_lineage.pkl.
.2 FUSE/.
.2 INTRAREG/.
.2 POST/.
.2 RAWDATA/.
.2 README.
.2 SEG/.
.2 parameters/.
}
\mbox{}

The \texttt{POST/} directory contains

\mbox{}
\dirtree{%
.1 POST/.
.2 POST\_RELEASE/.
.3 2019-Tutorial100\_post\_lineage.pkl.
.3 2019-Tutorial100\_post\_lineage.test.
.3 2019-Tutorial100\_post\_t000.inr.
.3 \ldots.
.3 2019-Tutorial100\_post\_t020.inr.
.3 5-postcorrection-tutorial-parameters.py.
.3 5-postcorrection.log.
}
\mbox{}



\section{Sequence intra-registration (or drift compensation) [3]}
\label{sec:tutorial:intra:registration:post}
 
We assume that we are located in the directory
\texttt{/path/to/astec/tutorial/tuto-astec1/}. Running the sequence intra-registration is
done with
\begin{code}{0.8}
  \$ 1.5-intraregistration.py -p parameters/1.5-intraregistration-tutorial-parameters-post.py 
\end{code}
\texttt{1.5-intraregistration-tutorial-parameters-post.py} being the
dedicated parameter file  (figure \ref{fig:tutorial:parameter:intra:registration:post}).

\begin{figure}
\begin{framed}
\begin{verbatim}
     1	PATH_EMBRYO = '.'
     2	
     3	EN = '2019-Tutorial100'
     4	
     5	begin = 0
     6	end = 20
     7	
     8	EXP_INTRAREG = 'POST'
     9	
    10	intra_registration_template_type = "POST-SEGMENTATION"
    11	intra_registration_template_threshold = 2
    12	intra_registration_margin = 20
    13	
    14	intra_registration_resample_post_segmentation_images = True
    15	intra_registration_resample_segmentation_images = True
    16	intra_registration_movie_post_segmentation_images = True
    17	intra_registration_movie_segmentation_images = True
\end{verbatim}
\end{framed}
\caption{\label{fig:tutorial:parameter:intra:registration:post} Tutorial
  parameter file for the sequence intra-registration step,
  post-segmentation images being used to build the template.}
\end{figure}

\begin{itemize}
  \itemsep -1ex
  \item The variable \texttt{PATH\_EMBRYO} is the path to the directory where
    the directory \texttt{FUSE/} is located. It can be either relative (as in the
    above example) or
    global (it could have been \texttt{/path/to/astec/tutorial/tuto-astec1/}).
  \item The variable \texttt{EN} is the prefix after which the  images
    are named. 
  \item The variables \texttt{begin} and \texttt{end} set respectively the
    first and the last index of the input time points to be processed.
  \item  the variable \texttt{EXP\_INTRAREG} set the suffix of the
    sub-directory of the \texttt{INTRAREG/} directory to be created.
  \item  the variable \texttt{intra\_registration\_template\_type} set
    the images to be used to build the template. Here, since it is
    equal to \texttt{"POST-SEGMENTATION''}, they are the
    post-corrected segmentation images obtained at the previous step.

    The variable \texttt{intra\_registration\_template\_threshold} set
    a threshold to be applied to the template images to define the
    information to be kept: we want all the points with a value equal
    or greater than 2 to be contained in the template after
    resampling. Since cells are labeled from 2 and above, the template
    is designed to contain all labeled cells after resampling, so it
    is built as small as possible.

    The variable \texttt{intra\_registration\_margin} allows to add
    margins (in the 3 dimensions) to the built template.
    
  \item The variable
    \texttt{intra\_registration\_resample\_post\_segmentation\_images}
    indicates whether the post-corrected segmentation images are to be resampled in
    the template geometry. 
  \item The variable
    \texttt{intra\_registration\_resample\_segmentation\_images}
    indicates whether the segmentation images are to be resampled in
    the template geometry. 
  \item The variable
    \texttt{intra\_registration\_movie\_post\_segmentation\_images}
    indicates whether 2D+t movies have to be built from the resampled
    post-corrected segmentation images.
  \item The variable
    \texttt{intra\_registration\_movie\_segmentation\_images}
    indicates whether 2D+t movies have to be built from the resampled
    segmentation images.
\end{itemize}

After processing, a \texttt{INTRAREG/INTRAREG\_POST/} directory has
been created and the \texttt{INTRAREG/} directory now contains

\mbox{}
\dirtree{%
.1 INTRAREG/.
.2 INTRAREG\_POST/.
.3 CO-TRSFS/.
.4 2019-Tutorial100\_intrareg\_flo000\_ref001.trsf.
.4 \ldots.
.4 2019-Tutorial100\_intrareg\_flo019\_ref020.trsf.
.3 FUSE/.
.4 2019-Tutorial100\_intrareg\_fuse\_t000.inr.
.4 \ldots.
.4 2019-Tutorial100\_intrareg\_fuse\_t020.inr.
.3 LOGS/.
.3 MOVIES/.
.4 2019-Tutorial100\_intrareg\_fuse\_t0-20\_xy174.inr.
.4 2019-Tutorial100\_intrareg\_post\_t0-20\_xy174.inr.
.4 2019-Tutorial100\_intrareg\_seg\_t0-20\_xy174.inr.
.3 POST/.
.4 2019-Tutorial100\_intrareg\_post\_t000.inr.
.4 \ldots.
.4 2019-Tutorial100\_intrareg\_post\_t020.inr.
.3 SEG/.
.4 2019-Tutorial100\_intrareg\_seg\_t000.inr.
.4 \ldots.
.4 2019-Tutorial100\_intrareg\_seg\_t020.inr.
.3 TRSFS\_t0-20/.
.4 2019-Tutorial100\_intrareg\_t000.trsf.
.4 \ldots.
.4 2019-Tutorial100\_intrareg\_t020.trsf.
.4 template\_t0-20.inr.
.2 INTRAREG\_RELEASE/.
.3 \ldots.
.2 INTRAREG\_SEG/.
.3 \ldots.
}
\mbox{}

In addition to directories already described in section
\ref{sec:tutorial:intra:registration:fuse}, the \texttt{INTRAREG\_POST/} directory contains
\begin{itemize}
\itemsep -1ex
\item The directory \texttt{POST/} contains the resampled
  post-corrected segmentation images
  in the same geometry (images have the same dimensions along X, Y and
  Z), with drift compensation (the eventual motion of the sample under the
  microscope has been compensated). 
\item In addition to a 2D+t movie made from the resampled fusion
  and the segmentation images, the directory \texttt{MOVIES/} contains
  a 2D+t movie made from the resampled post-corrected segmentation
  images.
\item The template image \texttt{template\_t0-20.inr} in the directory
  \texttt{TRSFS/} 
  is now of size $323 \times 265 \times 348$ with a voxel size of 0.6,
  has the same size than the one computed in section
  \ref{sec:tutorial:intra:registration:seg},  which is expected since
  the post-correction does not change the background.
  Note that all resampled images (in the \texttt{FUSE/}, the
  \texttt{POST/}, and the
  \texttt{SEG/} directories have the same geometry than the template
  image. 
\end{itemize}



\section{Sequence properties computation [2]}
\label{sec:tutorial:properties:post}

We assume that we are located in the directory
\texttt{/path/to/astec/tutorial/tuto-astec1/}.
Computing cell properties as well as lineage assumes that
 segmentation or post-corrected segmentation (see section
\ref{sec:tutorial:properties:post}) images have been
co-registered (see sections \ref{sec:tutorial:intra:registration:seg}
and \ref{sec:tutorial:intra:registration:post}). 
Extracting the sequence properties from the co-registered segmentation
images is
done with
\begin{code}{0.8}
  \$ X-embryoproperties.py -p parameters/X-embryoproperties-tutorial-parameters-post.py
\end{code}
\texttt{X-embryoproperties-tutorial-parameters-post.py} being the
dedicated parameter file  (figure \ref{fig:tutorial:parameter:properties:post}).

\begin{figure}
\begin{framed}
\begin{verbatim}
     1	PATH_EMBRYO = '.'
     2	
     3	EN = '2019-Tutorial100'
     4	
     5	begin = 0
     6	end = 20
     7	
     8	EXP_INTRAREG = 'POST'
\end{verbatim}
\end{framed}
\caption{\label{fig:tutorial:parameter:properties:post} Tutorial
  parameter file for the sequence properties from the co-registered
  post-corrected segmentation images.}
\end{figure}

\begin{itemize}
  \itemsep -1ex
  \item The variable \texttt{PATH\_EMBRYO} is the path to the directory where
    the directory \texttt{FUSE/} is located. It can be either relative (as in the
    above example) or
    global (it could have been \texttt{/path/to/astec/tutorial/tuto-astec1/}).
  \item The variable \texttt{EN} is the prefix after which the  images
    are named. 
  \item The variables \texttt{begin} and \texttt{end} set respectively the
    first and the last index of the input time points to be processed.
  \item  the variable \texttt{EXP\_INTRAREG} set the suffix of the
    sub-directory of the \texttt{INTRAREG/} directory where to search
    post-corrected segmentation or segmentation images.

    Since the directory \texttt{INTRAREG/INTRAREG\_POST/} does only
    contains the co-registered post-corrected segmentation images (in the
    \texttt{POST/} sub-directory), properties will be computed from
    these images preferably to the co-registered segmentation images (in the
    \texttt{SEG/} sub-directory).
\end{itemize}

After processing, some files appears in the \texttt{INTRAREG/INTRAREG\_POST/POST/} sub-directory 

\mbox{}
\dirtree{%
.1 INTRAREG/.
.2 INTRAREG\_RELEASE/.
.3 \ldots.
.2 INTRAREG\_SEG/.
.3 CO-TRSFS/.
.4 \ldots.
.3 FUSE/.
.4 \ldots.
.3 LOGS/.
.3 MOVIES/.
.4 \ldots.
.3 POST/.
.4 2019-Tutorial100\_intrareg\_post\_lineage.pkl.
.4 2019-Tutorial100\_intrareg\_post\_lineage.txt.
.4 2019-Tutorial100\_intrareg\_post\_lineage.xml.
.4 2019-Tutorial100\_intrareg\_post\_t000.inr.
.4 \ldots.
.4 2019-Tutorial100\_intrareg\_post\_t020.inr.
.3 SEG/.
.4 2019-Tutorial100\_intrareg\_seg\_t000.inr.
.4 \ldots.
.4 2019-Tutorial100\_intrareg\_seg\_t020.inr.
.3 TRSFS\_t0-20/.
.4 \ldots.
}
\mbox{}

Those files have the same meaning than the ones already presented in
section \ref{sec:tutorial:properties:seg}.
