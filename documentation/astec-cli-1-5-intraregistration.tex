\section{\texttt{1.5-intraregistration.py}}
\label{sec:cli:intraregistration}

The sequence intra-registration procedure can be done either after the fusion step, or after the (post-)segmentation step. It aims at
\begin{itemize}
\itemsep -1ex
\item compensating for the eventual motion of the imaged sample with respect to the microscope
\item resampling the fusion and/or the segmentation images into a common frame/geometry, so they can better be compared, and
\item building 2D+t images made of 2D sections from either the  fusion
  and/or the segmentation images, so that the quality of the fusion
  and/of the tracking step can be visually assessed. 
\end{itemize}


The intra-registration procedure is made of the following steps:
\begin{enumerate}
\itemsep -1ex
\item Co-registration of pairs of successive fused images (section
  \ref{sec:cli:intraregistration:coregistration}). This yields the
  transformations $T_{t+1 \leftarrow t}$. Fused images are located in
  \verb|<EMBRYO>/FUSE/FUSE_<EXP_FUSE>|: the parameter \verb|EXP_FUSE|
  is either set in the parameter file or is set at
  \verb|RELEASE|. This step may be long. 

\item Composition of transformations issued from the co-registration
  step. This step computes the transformations $T_{ref \leftarrow
    t}$. towards a reference image \verb|ref| given by the parameter
  \verb|intra_registration_reference_index|. 

\item Computation of the \textit{template} image (section
  \ref{sec:cli:intraregistration:template}). This \textit{template}
  image dimension are computed so that the useful information of all
  resampled images fits into it. Useful information can be issued from
  either the fused sequence, the segmentation sequence or the
  post-segmentation sequence. It is indicated by the
  \verb|intra_registration_template_type| which value can be either
  \verb|'FUSION'|,  \verb|'SEGMENTATION'|, or
  \verb|'POST-SEGMENTATION'|. This step may be long. 

\item Resampling of either the fused or the segmentation images
  (section \ref{sec:cli:intraregistration:resampling}). Note that
  changing the parameters for this step will not require to re-compute
  the first steps. 

\item Extraction of 2D+t images from the resampled sequences (section
  \ref{sec:cli:intraregistration:movies}). Note that changing the
  parameters for this step (i.e. requiring extra movies) will not
  require to re-compute the first steps, with an eventual exception
  for the resampling step. 

\end{enumerate}



\subsection{\texttt{1.5-intraregistration.py} options}

The following options are available:
\begin{description}
  \itemsep -1ex
\item[\texttt{-h}] prints a help message
\item[\texttt{-p \underline{file}}] set the parameter file to be parsed
\item[\texttt{-e \underline{path}}] set the
  \texttt{\underline{path}} to the directory where the
  \texttt{RAWDATA/} directory is located
\item[\texttt{-k}] allows to keep the temporary files
\item[\texttt{-f}] forces execution, even if (temporary) result files
  are already existing
\item[\texttt{-v}] increases verboseness (both at console and in the
  log file)
\item[\texttt{-nv}] no verboseness
\item[\texttt{-d}]  increases debug information (in the
  log file)
\item[\texttt{-nd}] no debug information
\end{description}


\subsection{Output data}

The results are stored in sub-directories
\texttt{INTRAREG/INTRAREG\_<EXP\_INTRAREG>} under the
\texttt{/path/to/experiment/} directory where \texttt{<EXP\_INTRAREG>} is the value of the variable \texttt{EXP\_INTRAREG} (its
default value is '\texttt{RELEASE}'). 

\dirtree{%
.1 /path/to/experiment/.
.2 \ldots.
.2 INTRAREG/.
.3 INTRAREG\_<EXP\_INTRAREG>/.
.4 CO-TRSFS/.
.4 FUSE/.
.4 LOGS/.
.4 MOVIES/.
.4 [POST/].
.4 [SEG/].
.4 TRSFS\_t<begin>-<end>/.
.2 \ldots.
}


\texttt{INTRAREG/INTRAREG\_<EXP\_INTRAREG>/FUSE} contains the fused
images after resampling in a common geometry (i.e. all image have the
same X, Y and Z dimensions)

\dirtree{%
.1 /path/to/experiment/.
.2 \ldots.
.2 INTRAREG/.
.3 INTRAREG\_<EXP\_INTRAREG>/.
.4 \ldots.
.4 FUSE/.
.5 <EN>\_intrareg\_fuse\_t<xxx>.inr
.4 \ldots.
.2 \ldots.
}



\subsection{Co-registration parameters}
\label{sec:cli:intraregistration:coregistration}
Default registration parameters for the co-registration are set by:
\begin{verbatim}
# intra_registration_compute_registration = True
# intra_registration_transformation_type = 'rigid'
# intra_registration_transformation_estimation_type = 'wlts'
# intra_registration_lts_fraction = 0.55
# intra_registration_pyramid_highest_level = 6
# intra_registration_pyramid_lowest_level = 3
# intra_registration_normalization = True
\end{verbatim}
Computed transformations are stored in \verb|INTRAREG/INTRAREG_<EXP>/CO-TRSFS|.
It may be advised to set the pyramid lowest level value to some higher value to speed up the co-registrations (recall that all pairs of successive images will be co-registered, i.e.
\begin{verbatim}
intra_registration_pyramid_lowest_level = 4
\end{verbatim}


\subsection{Template building parameters}
\label{sec:cli:intraregistration:template}

\begin{verbatim}
# intra_registration_reference_index = None
# intra_registration_template_type = 'FUSION'
# intra_registration_template_threshold = None
# intra_registration_resolution = 0.6
# intra_registration_margin = None
\end{verbatim}

The \verb|intra_registration_reference_index| allows to choose the reference image (the one which remains still, i.e. is only displaced by a translation), by default it is the first image image of the series (associated to \verb|begin|).

Depending on \verb|intra_registration_template_type| (\verb|'FUSION'|, \verb|'SEGMENTATION'| or \verb|'POST-SEGMENTATION'|, the two latter assume obviously that the segmentation has been done), the \textit{template} image can be built either after the fusion or the segmentation images. If no threshold is given by \verb|intra_registration_template_threshold|, the built template will be large enough to include all the transformed fields of view (in this case, the template is the same whatever \verb|intra_registration_template_type| is). 

If a threshold is given, the built template will be large enough to
include all the transformed points above the threshold. E.g., the
background is labeled with either '1' or '0' in segmentation images,
then a threshold of '2' ensures that all the embryo cells will not be
cut by the resampling stage.  In this case, adding an additional
margin to the template could be a good idea for visualization
purpose. Last but not least, using a larger resolution than the
\verb|target_resolution| (the resolution of the fused images) allows
to decrease the resampled images volume. This can be achieved by
setting \verb|intra_registration_resolution| to a larger value than
the one of \verb|target_resolution| (default is 0.6).

Thus, building a \textit{template} image after the segmentation images can be done with
\begin{verbatim}
# intra_registration_reference_index = None
intra_registration_template_type = "SEGMENTATION"
intra_registration_template_threshold = 2
# intra_registration_resolution = 0.6
intra_registration_margin = 10
\end{verbatim}

Computed transformations from the \textit{template} image as well as the \textit{template} image itself are stored in \verb|INTRAREG/INTRAREG<EXP>/TRSFS_t<F>-<L>/| where \verb|<F>| and \verb|L| are the first and the last index of the series (specified by \verb|begin| and \verb|end| from the parameter file).

\subsubsection{Resampling fusion/segmentation images}
\label{sec:cli:intraregistration:resampling}
The resampled fusion and segmentation images will be stored respectively in \verb|INTRAREG/INTRAREG_<EXP>/FUSE|, \verb|INTRAREG/INTRAREG_<EXP>/SEG/| and \verb|INTRAREG/INTRAREG_<EXP>/POST/|. Resampling is done either if the following parameters are set to \verb|True| or if movies are requested to be computed.

\begin{verbatim}
# intra_registration_resample_fusion_images = True
# intra_registration_resample_segmentation_images = False
# intra_registration_resample_post_segmentation_images = False
\end{verbatim}

\subsection{2D+t movies}
\label{sec:cli:intraregistration:movies}
For either visual assessment or illustration purposes, 2D+t (i.e. 3D) images can be built from 2D sections extracted from the resampled temporal series. This is controlled by the following parameters:
\begin{verbatim}
# intra_registration_movie_fusion_images = True
# intra_registration_movie_segmentation_images = False
# intra_registration_movie_post_segmentation_images = False

# intra_registration_xy_movie_fusion_images = [];
# intra_registration_xz_movie_fusion_images = [];
# intra_registration_yz_movie_fusion_images = [];

# intra_registration_xy_movie_segmentation_images = [];
# intra_registration_xz_movie_segmentation_images = [];
# intra_registration_yz_movie_segmentation_images = [];

# intra_registration_xy_movie_post_segmentation_images = [];
# intra_registration_xz_movie_post_segmentation_images = [];
# intra_registration_yz_movie_post_segmentation_images = [];
\end{verbatim}

If \verb|intra_registration_movie_fusion_images| is set to \verb|True|, a movie is made with the  XY-section located at the middle of each resampled fusion image (recall that, after resampling, all images have the same geometry). Additional XY-movies can be done by specifying the wanted Z values in \verb|intra_registration_xy_movie_fusion_images|. E.g.
\begin{verbatim}
intra_registration_xy_movie_fusion_images = [100, 200];
\end{verbatim}
will build two movies with XY-sections located respectively at Z values of 100 and 200. The same stands for the other orientation and for the resampled segmentation images.

\subsection{Parameter list}

Please also refer to the file
\texttt{parameter-file-examples/1.5-intraregistration-parameters.py}

\begin{itemize}
\itemsep -1ex
\item \texttt{EN}
\item \texttt{EXP\_FUSE}
\item \texttt{EXP\_INTRAREG}
\item \texttt{EXP\_POST}
\item \texttt{EXP\_SEG}
\item \texttt{PATH\_EMBRYO}
\item \texttt{begin}
\item \texttt{default\_image\_suffix}
\item \texttt{delta}
\item \texttt{end}
\item \texttt{intra\_registration\_compute\_registration}
\item \texttt{intra\_registration\_lts\_fraction}
\item \texttt{intra\_registration\_margin}
\item \texttt{intra\_registration\_movie\_fusion\_images}
\item \texttt{intra\_registration\_movie\_post\_segmentation\_images}
\item \texttt{intra\_registration\_movie\_segmentation\_images}
\item \texttt{intra\_registration\_normalization}
\item \texttt{intra\_registration\_pyramid\_highest\_level}
\item \texttt{intra\_registration\_pyramid\_lowest\_level}
\item \texttt{intra\_registration\_reference\_index}
\item \texttt{intra\_registration\_resample\_fusion\_images}
\item \texttt{intra\_registration\_resample\_post\_segmentation\_images}
\item \texttt{intra\_registration\_resample\_segmentation\_images}
\item \texttt{intra\_registration\_resolution}
\item \texttt{intra\_registration\_sigma\_segmentation\_images}
\item \texttt{intra\_registration\_template\_threshold}
\item \texttt{intra\_registration\_template\_type}
\item \texttt{intra\_registration\_transformation\_estimation\_type}
\item \texttt{intra\_registration\_transformation\_type}
\item \texttt{intra\_registration\_xy\_movie\_fusion\_images}
\item \texttt{intra\_registration\_xy\_movie\_post\_segmentation\_images}
\item \texttt{intra\_registration\_xy\_movie\_segmentation\_images}
\item \texttt{intra\_registration\_xz\_movie\_fusion\_images}
\item \texttt{intra\_registration\_xz\_movie\_post\_segmentation\_images}
\item \texttt{intra\_registration\_xz\_movie\_segmentation\_images}
\item \texttt{intra\_registration\_yz\_movie\_fusion\_images}
\item \texttt{intra\_registration\_yz\_movie\_post\_segmentation\_images}
\item \texttt{intra\_registration\_yz\_movie\_segmentation\_images}
\item \texttt{result\_image\_suffix}
\end{itemize}
